% Исследовательская часть
\section{Исследовательская часть}

\hspace{1.25cm}
Для сравнения времени реализаций алгоритмов Левенштейна и Дамерау-Левенштейна в их матричных, рекурсивных и рекурсивно-матричных реализациях программа была запущена на рандомно сгенерированных строках длинами от 1 до 9 с шагом 2 по 50 замеров каждая строка, среднее значение было вынесено в таблицу и для наглядности изображено на графике (рисунок~\ref{fig:graph_all} и таблица~\ref{table:table_all})

\begin{figure}[H]
    \centering
    \includegraphics[width=1\textwidth]{img/graph_all.png}
    \caption{График времени работы всех алгоритмов в зависимости от длин строк}
    \label{fig:graph_all} % Метка для ссылки на картинку
\end{figure}

\begin{table}[H]
    \centering
    \begin{tabular}{|l|c|c|c|c|c|c|c|c|}
        \hline
        \textbf{Алгоритм} & \textbf{1} & \textbf{2} & \textbf{3} & \textbf{4} & \textbf{5} & \textbf{6} & \textbf{7} & \textbf{8}\\
        \hline
        Левенштейна (матричный) & 0.0 & 0.0 & 0.0 & 0.0 & 0.0 & 0.0 & 0.0& 0.0 \\
        Левенштейна (рекурсивный) & 0.0 & 0.0 & 0.0 & 0.0 & 0.0 & 0.0022 & 0.014 & 0.056 \\
        Левенштейна (рекурс-мат.) & 0.0 & 0.0 & 0.0 & 0.0 & 0.0 & 0.0 & 0.0 & 0.0 \\
        Дамерау-Левенштейна (матричный) & 0.0 & 0.0 & 0.0 & 0.0 & 0.0 & 0.0 & 0.0 & 0.0 \\
        Дамерау-Левенштейна (рекурсивный) & 0.0 & 0.0 & 0.0 & 0.0 & 0.0 & 0.0019 & 0.0097 & 0.054 \\
        Дамерау-Левенштейна (рекурс-мат.) & 0.0 & 0.0 & 0.0 & 0.0 & 0.0 & 0.0 & 0.0 & 0.0 \\
        \hline
    \end{tabular}
    \caption{Таблица времени (сек) работы всех алгоритмов в зависимости от длин строк (сим)}
    \label{table:table_all}
\end{table}

Кроме того, замеры времени работы всех алгоритмов были проведены на микроконтроллерах STM32. На графике~\ref{fig:graph_all_micro} и в таблице~\ref{table:table_all_micro} приведены замеры на длинах строк от 1 символа до 6 с шагом 1 и запуском каждого по 10 раз.

\begin{figure}[H]
    \centering
    \includegraphics[width=1\textwidth]{img/graph_all_micro.png}
    \caption{График времени работы всех алгоритмов в зависимости от длин строк (замеры на микроконтроллерах)}
    \label{fig:graph_all_micro} % Метка для ссылки на картинку
\end{figure}

\begin{table}[H]
    \centering
    \begin{tabular}{|l|c|c|c|c|c|c|}
        \hline
        \textbf{Алгоритм} & \textbf{1} & \textbf{2} & \textbf{3} & \textbf{4} & \textbf{5} & \textbf{6}\\
        \hline
        Левенштейна (матричный) & 0.0001 & 0.0002 & 0.0003 & 0.0005 & 0.0007 & 0.0009 \\
        Левенштейна (рекурсивный) & 0.0001 & 0.0005 & 0.0026 & 0.0138 & 0.0732 & 0.3941 \\
        Левенштейна (рекурс-мат.) & 0.0003 & 0.0007 & 0.0015 & 0.0026 & 0.0043 & 0.0060 \\
        Дамерау-Левенштейна (матрич.) & 0.0000 & 0.0002 & 0.0004 & 0.0006 & 0.0009 & 0.00013 \\
        Дамерау-Левенштейна (рекурс.) & 0.0001 & 0.0006 & 0.0029 & 0.0151 & 0.0801 & 0.4233 \\
        Дамерау-Левенштейна (рек-мат.) & 0.0003 & 0.0007 & 0.0015 & 0.0029 & 0.0045 & 0.0065 \\
        \hline
    \end{tabular}
    \caption{Таблица времени (сек) работы всех алгоритмов в зависимости от длин строк (сим) (замеры на микроконтроллерах)}
    \label{table:table_all_micro}
\end{table}

\subsection{Сравнение работы матричной, рекурсивной и рекурсивно-матричной реализаций алгоритмов}

\hspace{1.25cm}
Из графиков, приведённых выше, очевидно, что матричная реализация обоих алгоритмов быстро становится эффективнее рекурсивной на много порядков. Это происходит из-за того, что при рекурсии даже на небольшой длине строк происходит много рекурсивных вызовов для подстрок, на что тратится большое количество времени и памяти. В то время как для матричной реализации данные, на основе которых вычисляются следующие значения, хранятся в двух массивах длинной в кратчайшую из двух строк, что экономит как время, так и память. При этом рекурсивно-матричная реализация оказалась почти столь же быстрой, как и матричная благодаря исключению повторных вычислений идентичных веток рекурсии, что в разы сократило количество вычислений.

\subsection{Сравнение работы алгоритмов Левенштейна и Дамерау-Левенштейна (отдельно каждый способ)}

\hspace{1.25cm}
Отдельно было измерено время работы алгоритмов Левенштейна и Дамерау-Левенштейна в их матричных реализациях на большем диапазоне длин рандомно сгенерированных строк (от 25 символов до 125 с шагом 25) по 50 замеров каждая строка, среднее значение было вынесено в таблицу и для наглядности изображено на графике (рисунок~\ref{fig:graph_mat} и таблица~\ref{table:table_mat}).

\begin{figure}[H]
    \centering
    \includegraphics[width=1\textwidth]{img/graph_mat.png}
    \caption{График времени работы матричных реализаций алгоритмов в зависимости от длин строк}
    \label{fig:graph_mat}
\end{figure}

\begin{table}[H]
    \centering
    \begin{tabular}{|l|c|c|c|c|c|}
        \hline
        \textbf{Алгоритм} & \textbf{25} & \textbf{50} & \textbf{75} & \textbf{100} & \textbf{125}\\
        \hline
        Левенштейна (матричный) & 0.0 & 0.0 & 0.0 & 0.0013 & 0.0016 \\
        Дамерау-Левенштейна (матричный) & 0.0 & 0.00031 & 0.0016  & 0.0022 & 0.0028 \\
        \hline
    \end{tabular}
    \caption{Таблица времени (сек) работы матричных реализаций алгоритмов в зависимости от длин строк (сим)}
    \label{table:table_mat}
\end{table}

Отдельно было измерено время работы алгоритмов Левенштейна и Дамерау-Левенштейна в их рекурсивных реализациях на малом диапазоне длин рандомно сгенерированных строк (от 1 символа до 9 с шагом 2) по 50 замеров каждая строка, среднее значение было вынесено в таблицу и для наглядности изображено на графике (рисунок~\ref{fig:graph_rec} и таблица~\ref{table:table_rec}).

\begin{figure}[H]
    \centering
    \includegraphics[width=1\textwidth]{img/graph_rec.png}
    \caption{График времени работы рекурсивных реализаций алгоритмов в зависимости от длин строк}
    \label{fig:graph_rec}
\end{figure}

\begin{table}[H]
    \centering
    \begin{tabular}{|l|c|c|c|c|c|}
        \hline
        \textbf{Алгоритм} & \textbf{1} & \textbf{3} & \textbf{5} & \textbf{7} & \textbf{9}\\
        \hline
        Левенштейна (рекурсивный) & 0.0 & 0.0 & 0.00031 & 0.012 & 0.39 \\
        Дамерау-Левенштейна (рекурсивный) &  0.0 & 0.0 & 0.0 & 0.034 & 0.66  \\
        \hline
    \end{tabular}
    \caption{Таблица времени (сек) работы рекурсивных реализаций алгоритмов в зависимости от длин строк (сим)}
    \label{table:table_rec}
\end{table}

Отдельно было измерено время работы алгоритмов Левенштейна и Дамерау-Левенштейна в их рекурсивно-матричных реализациях на большем диапазоне длин рандомно сгенерированных строк (от 25 символа до 125 с шагом 25) по 50 замеров каждая строка, среднее значение было вынесено в таблицу и для наглядности изображено на графике (рисунок~\ref{fig:graph_rec-mat} и таблица~\ref{table:table_rec-mat}).

\begin{figure}[H]
    \centering
    \includegraphics[width=1\textwidth]{img/graph_rec-mat.png}
    \caption{График времени работы рекурсивно-матричных реализаций алгоритмов в зависимости от длин строк}
    \label{fig:graph_rec-mat}
\end{figure}

\begin{table}[H]
    \centering
    \begin{tabular}{|l|c|c|c|c|c|}
        \hline
        \textbf{Алгоритм} & \textbf{25} & \textbf{50} & \textbf{75} & \textbf{100} & \textbf{125}\\
        \hline
        Левенштейна (рек-мат.) & 0.00063 & 0.0013 & 0.0016 & 0.0041 & 0.0094 \\
        Дамерау-Левенштейна (рек-мат.) & 0.0 & 0.0025 & 0.0028 & 0.0053 & 0.013  \\
        \hline
    \end{tabular}
    \caption{Таблица времени (сек) работы рекурсивно-матричных реализаций алгоритмов в зависимости от длин строк (сим)}
    \label{table:table_rec-mat}
\end{table}

Видно, что алгоритм Левенштейна оказался немного быстрее алгоритма Дамерау-Левенштейна из-за дополнительной проверки во втором, что компенсируется большим расстоянием в результате первого при наличии перестановок букв в строках.

\subsection{Сравнение работы матричных и рекурсивно-матричных алгоритмов Левенштейна и Дамерау-Левенштейна}

\hspace{1.25cm}
Так как на общем графике матричный и рекурсивно-матричный алгоритмы были очень близки по скорости, были проведены отдельные замеры (на 5-и точках с длиной строк от 25 до 125 символов с шагом 25 по 50 запусков), среднее значение было вынесено в таблицу и для наглядности изображено на графике (рисунок~\ref{fig:graph_mat_rec-mat} и таблица~\ref{table:table_mat_rec-mat}).

\begin{figure}[H]
    \centering
    \includegraphics[width=1\textwidth]{img/graph_mat_rec-mat.png}
    \caption{График времени работы матричных и рекурсивно-матричных реализаций алгоритмов в зависимости от длин строк}
    \label{fig:graph_mat_rec-mat}
\end{figure}

\begin{table}[H]
    \centering
    \begin{tabular}{|l|c|c|c|c|c|}
        \hline
        \textbf{Алгоритм} & \textbf{25} & \textbf{50} & \textbf{75} & \textbf{100} & \textbf{125}\\
        \hline
        Левенштейна (матричный) & 0.0 & 0.00063 & 0.00094 & 0.00094 & 0.00094 \\
        Левенштейна (рек-мат.) & 0.00031 & 0.0016 & 0.0025 & 0.0069 & 0.011 \\
        Дамерау-Левенштейна (матричный) & 0.0 & 0.00031 & 0.00094 & 0.0019 & 0.0013 \\
        Дамерау-Левенштейна (рек-мат.) & 0.00031 & 0.0022 & 0.005 & 0.0075 & 0.014 \\
        \hline
    \end{tabular}
    \caption{Таблица времени (сек) работы матричных и рекурсивно-матричных реализаций алгоритмов в зависимости от длин строк (сим)}
    \label{table:table_mat_rec-mat}
\end{table}

Кроме того, замеры времени работы матричных и рекурсивно-матричных алгоритмов были проведены на микроконтроллерах STM32. На графике~\ref{fig:graph_mat_rec-mat_micro} и в таблице~\ref{table:table_mat_rec-mat_micro} приведены замеры на длинах строк от 5 символов до 45 с шагом 5 и запуском каждого по 20 раз.

\begin{figure}[H]
    \centering
    \includegraphics[width=1\textwidth]{img/graph_mat_rec-mat_micro.png}
    \caption{График времени работы матричных и рекурсивно-матричных реализаций алгоритмов в зависимости от длин строк (замеры на микроконтроллерах)}
    \label{fig:graph_mat_rec-mat_micro} % Метка для ссылки на картинку
\end{figure}

\begin{table}[H]
    \centering
    \begin{tabular}{|l|c|c|c|c|c|c|c|c|c|}
        \hline
        \textbf{Алгоритм} & \textbf{5} & \textbf{10} & \textbf{15} & \textbf{20} & \textbf{25} & \textbf{30} & \textbf{35} & \textbf{40} & \textbf{45} \\
        \hline
        Левенш(м) & 0.0007 & 0.0025 & 0.0053 & 0.0095 & 0.0146 & 0.0214 & 0.0292 & 0.0385 & 0.0492 \\
        Левенш(р-м) & 0.0042 & 0.0171 & 0.0394 & 0.0715 & 0.1138 & 0.1652 & 0.2277 & 0.3029 & 0.3876 \\
        Дам-Лев(м) & 0.0009 & 0.0036 & 0.0081 & 0.0143 & 0.0228 & 0.0335 & 0.0458 & 0.0607 & 0.0782 \\
        Дам-Лев(р-м) & 0.0045 & 0.0187 & 0.0431 & 0.0783 & 0.1244 & 0.1810 & 0.2505 & 0.3317 & 0.4253 \\
        \hline
    \end{tabular}
    \caption{Таблица времени (сек) работы матричных и рекурсивно-матричных реализаций алгоритмов в зависимости от длин строк (сим) (замеры на микроконтроллерах)}
    \label{table:table_mat_rec-mat_micro}
\end{table}

По результатам приведённых графиков видно, что и в алгоритме Левенштейна и Дамерау-Левенштейна рекурсивно-матричный метод работает дольше матричного. Это объясняется затратами на вызов функции при рекурсии и на дополнительные проверки является ли искомое значение уже посчитанным. На небольших длинах строк разница в скорости работы алгоритмов отличается несущественно, однако с увеличением данных растёт и разница во времени.

\subsection*{Вывод}

\hspace{1.25cm}
По проведённым исследованиям была выявлена большая скорость работы алгоритма Левенштейна над алгоритмом Дамерау-Левенштейна за счёт уменьшения числа проверок, что, однако, даёт иной результат при наличии возможности перестановок символов в строках. При этом матричный вариант выигрывает по скорости в обоих алгоритмах, на втором месте оказался рекурсивно-матричный метод, который делает меньше рекурсивных вызовов, чем рекурсивный метод, и исключает повторные вычисления идентичных веток, так как при вызове каждой новой функции в этом методе передаётся в качестве аргумента ссылка на матрицу, которая хранит уже посчитанные значения, но на эту матрицу также необходима память, а проверки на уже вычисленные значения не всегда приносят положительный результат и занимают время.

\newpage