% Введение
% Центрируем заголовок и делаем его капсом
\begin{center}
    \MakeUppercase{\large Введение}
\end{center}
\addcontentsline{toc}{section}{Введение} % Добавляем в оглавление

Цель лабораторной работы: исследовать алгоритмы вычисления расстояния Левенштейна и Дамерау-Левенштейна в матричной, рекурсивно-матричной и рекурсивной реализациях.

\vspace{0.25cm}
Для достижения этой цели были поставлены следующие задачи:

\begin{itemize}

\item рассмотреть алгоритм вычисления расстояния Левенштейна;

\item рассмотреть алгоритм вычисления расстояния Дамерау-Левенштейна;

\item применить метод динамического программирования для матричных реализаций алгоритмов;

\item сравнить матричную, рекурсивно-матричную и рекурсивную реализации алгоритмов;

\item сравнить алгоритмы вычисления расстояния Левенштейна и Дамерау-Левенштейна.

\end{itemize}

\newpage