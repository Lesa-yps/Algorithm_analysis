% Список использованных источников
\begin{center}
    \MakeUppercase{\large Список использованных источников}
\end{center}
\addcontentsline{toc}{section}{Список использованных источников} % Добавляем в оглавление

% Удаление "Список литературы" по умолчанию и получившихся пробелов
\renewcommand{\refname}{}
\vspace{-11mm}

\begin{thebibliography}{9}
    \bibitem{process_time} Python Documentation. \textit{time.process\_time() - Документация по стандартной библиотеке Python}. Дата обращения: 03 сентября 2024 г. [Электронный ресурс]. Доступно по адресу: \url{https://docs-python.ru/standart-library/modul-time-python/funktsija-process-time-modulja-time/}
    
    \bibitem{tirinox} Tirinox. \textit{Алгоритм Левенштейна на Python: реализация и объяснение}. Дата обращения: 02 сентября 2024 г. [Электронный ресурс]. Доступно по адресу: \url{https://tirinox.ru/levenstein-python/}
    
    \bibitem{gasfild} Гасфилд Дэн. \textit{Строки, деревья и последовательности в алгоритмах}: Информатика и вычислительная биология / Пер. с англ. И. В. Романовского. — СПб.: Невский Диалект; БХВ-Петербург, 2003. — 654 с: ил.
    
    \bibitem{levenshtein} Левенштейн В. И. \textit{Двоичные коды с исправлением выпадений, вставок и замещений символов}. Доклады Академий Наук СССР, 1965. 163.4:845-848.
    
    \bibitem{niezov} Ниёзов Д. Л. \textit{Применение методов нечеткого сравнения строк в прикладных задачах}: Выпускная квалификационная работа (Бакалаврская работа). — Тольятти: Тольяттинский государственный университет, 2020. — 45 стр.
    
\end{thebibliography}