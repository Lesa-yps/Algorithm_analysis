% Заключение
\begin{center}
    \MakeUppercase{\large Заключение}
\end{center}
\addcontentsline{toc}{section}{Заключение} % Добавляем в оглавление

В результате выполнения лабораторной работы были исследованы алгоритмы вычисления расстояния Левенштейна и Дамерау-Левенштейна в матричной, рекурсивно-матричной и рекурсивной реализациях.

\vspace{0.25cm}
В частности:

\begin{itemize}

\item были рассмотрены алгоритмы вычисления расстояния Левенштейна и Дамерау-Левенштейна;

\item применён метод динамического программирования для матричных реализаций алгоритмов;

\item сравнены матричная, рекурсивно-матричная и рекурсивная реализации алгоритмов;

\item сравнены алгоритмы вычисления расстояния Левенштейна и Дамерау-Левенштейна.

\end{itemize}

В ходе лабораторной работы были рассмотрены, спроетированны и запрограммированы алгоритмы нахождения расстояний Левенштейна и Дамерау-Левенштейна в их матричных, рекурсивных и рекурсивно-матричных реализациях.

Были разработаны тесты для всех алгоритмов, учитывающие крайние случаи, ожидаемых результатов которых достигли все реализации.

Сравнения полученных программ показали, что алгоритм Левенштейна работает быстрее алгоритма Дамерау-Левенштейна за счёт меньшего числа проверок, однако это приводит к другим результатам, если возможны перестановки символов. Матричный вариант оказался самым быстрым среди всех, на втором месте — рекурсивно-матричный метод, который снижает количество рекурсивных вызовов и избегает повторных вычислений, используя матрицу для хранения уже найденных значений. Однако этот метод требует дополнительной памяти, а проверки на уже вычисленные значения не всегда ускоряют процесс и также занимают время.

\newpage