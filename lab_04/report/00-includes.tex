% Пакеты и настройки

\setlength{\parindent}{1.25cm} % Установка абзацного отступа в 1.25 см

% Пакеты для поддержки кириллицы
\usepackage[utf8]{inputenc}    % Кодировка UTF-8
\usepackage{cmap} % Улучшенный поиск русских слов в полученном pdf-файле
\usepackage[T2A]{fontenc} % Поддержка русских букв
\usepackage[english,russian]{babel} % Русский и английский языки
\usepackage{mathptmx}          % Используем шрифт Times New Roman вместо Computer Modern

\usepackage{fancyhdr} % Для настройки колонтитулов (опционально)

\usepackage{amsmath,amsfonts,amssymb} % Математические символы
\usepackage{graphicx}          % Для вставки изображений
\usepackage{geometry}          % Настройка полей
\usepackage{float}             % Фиксация местоположения таблиц и рисунков
\usepackage{hyperref}          % Ссылки
\usepackage{titlesec}          % Для оформления заголовков разделов
\usepackage{xcolor}			  % Цвета

\usepackage{tabularx} % для таблиц

\usepackage{amsmath, amssymb, amsthm}  % Подключение пакетов для математики и теорем

% Определение стиля окружения "definition"
\newtheorem{definition}{Определение}

% Отключаем добавление библиографии и содержания в оглавление
\usepackage[notbib,nottoc]{tocbibind}


\usepackage{longtable} % Для длинных таблиц, если нужно
\usepackage{array}     % Для улучшенного контроля над таблицами
\usepackage{makecell} % Пакет для работы с многострочными ячейками

\usepackage{setspace} % Для междустрочных интервалов
\onehalfspacing % Полуторный интервал

\usepackage{hyperref} % Пакет для ссылок (в источниках)

% Настройка полей
\geometry{
    left=30mm,
    right=15mm,
    top=20mm,
    bottom=20mm
}

% Настройка стиля заголовков
\titleformat{\section}{\large\bfseries}{\thesection.}{1em}{}
\titleformat{\subsection}{\normalsize\bfseries}{\thesubsection.}{1em}{}

\usepackage{caption} % Для настройки подписей
\captionsetup[figure]{justification=centering} % Центрирование подписей к рисункам
% Изменяем текст "Рис." на "Рисунок"
\addto\captionsrussian{\renewcommand{\figurename}{Рисунок}}
% Настраиваем разделитель с ":" на " -"
\captionsetup[figure]{labelsep=endash} % Используем длинное тире для разделителя

% Настройка внешнего вида подписей таблиц
\captionsetup[table]{justification=centering, labelsep=endash} % Центрируем и используем тире вместо двоеточия

% Для кода
\usepackage{listings}
% Настройка стиля для кода
\lstset{
    language=Python,            % Язык кода
    basicstyle=\ttfamily\footnotesize,  % Базовый стиль текста и размер шрифта
    backgroundcolor=\color{gray!10},    % Цвет фона
    frame=single,              % Рамка вокруг листинга
    breaklines=true,           % Переносить строки
    captionpos=b,              % Позиция заголовка (caption) снизу
    tabsize=4,                 % Размер табуляции
    commentstyle=\color{gray}, % Стиль комментариев
    keywordstyle=\color{blue}, % Стиль ключевых слов
    stringstyle=\color{red},   % Стиль строк
    showstringspaces=false,    % Не отображать пробелы в строках
    numbers=left, % Нумерация строк слева
    numberstyle=\tiny\color{gray}, % Стиль нумерации строк
    stepnumber=1, % Шаг нумерации строк
    numbersep=5pt, % Расстояние между нумерацией и текстом
    breaklines=true, % Перенос длинных строк
    aboveskip=1em, % Расстояние выше листинга
    belowskip=1em, % Расстояние ниже листинга
}

% Изменяем разделитель в подписях
\captionsetup[lstlisting]{labelsep=endash} % Используем длинное тире

% Настройка стиля заголовков
\titleformat{\section}{\normalfont\Large\bfseries}{\thesection}{1em}{}
\titleformat{\subsection}{\normalfont\large\bfseries}{\thesubsection}{1em}{}