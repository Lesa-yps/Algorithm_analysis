% Введение
% Центрируем заголовок и делаем его заглавными буквами
\begin{center}
    \MakeUppercase{Введение}
\end{center}
\addcontentsline{toc}{section}{\MakeUppercase{Введение}} % Добавляем в оглавление с заглавными буквами

В современных вычислительных системах параллельные вычисления играют важную роль в повышении производительности программного обеспечения. Одним из основных методов организации параллелизма является использование нативных потоков операционной системы (OS threads), которые позволяют распределять выполнение задач между несколькими ядрами процессора.
\vspace{0.25cm}

Цель работы: исследовать особенности параллельных вычислений на основе нативных потоков и время работы программы на их разном количестве.

\vspace{0.25cm}
Для достижения этой цели были поставлены следующие задачи:

\begin{enumerate}

\item анализ предметной области;
\item разработка алгоритма обработки данных;
\item создание ПО реализующего разработанный алгоритм;
\item исследование характеристик созданного ПО.

\end{enumerate}

% Входные и выходные данные
\section{Входные и выходные данные}

\hspace{1.25cm}
Входными данными для ПО является адрес главной страницы ресурса. Выходные данные --- директория с файлами, которые содержат скачанные данные со страниц в формате, пригодном для дальнейшей обработки (html), а также время работы.

% Преобразование входных данных в выходные
\section{Преобразование входных данных в выходные}

\hspace{1.25cm}
Программа считывает URL-адреса из файла, загружает содержимое страниц и сохраняет их локально как файлы. Для каждого URL создаётся поток, который обрабатывает загрузку и запись данных. Имена файлов формируются на основе URL с заменой недопустимых символов.