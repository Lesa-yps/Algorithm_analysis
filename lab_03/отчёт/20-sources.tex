% Список использованных источников
\begin{center}
    \MakeUppercase{\large Список использованных источников}
\end{center}
\addcontentsline{toc}{section}{Список использованных источников} % Добавляем в оглавление

% Удаление "Список литературы" по умолчанию и получившихся пробелов
\renewcommand{\refname}{}
\vspace{-11mm}

\begin{thebibliography}{9}
    \bibitem{process_time} Python Documentation. \textit{time.process\_time() - Документация по стандартной библиотеке Python}. Дата обращения: 03 сентября 2024 г. [Электронный ресурс]. Доступно по адресу: \url{https://docs-python.ru/standart-library/modul-time-python/funktsija-process-time-modulja-time/}
    
    \bibitem{geeks} Geeksforgeeks: Поиск элементов в массиве | Операции с массивами. Дата обращения: 25 сентября 2024 г. [Электронный ресурс]. Доступно по адресу: \url{https://www.geeksforgeeks.org/searching-elements-in-an-array-array-operations/}
    
    \bibitem{geeks_linear} Geeksforgeeks: Введение в алгоритм линейного поиска. Дата обращения: 25 сентября 2024 г. [Электронный ресурс]. Доступно по адресу: \url{https://https://www.geeksforgeeks.org/linear-search/}
    
    \bibitem{geeks_binary} Geeksforgeeks: Алгоритм бинарного поиска – Итеративная и рекурсивная реализация. Дата обращения: 25 сентября 2024 г. [Электронный ресурс]. Доступно по адресу: \url{https://www.geeksforgeeks.org/binary-search/}
    
    \bibitem{lorin1983} Г. Лорин. \textit{Сортировка и системы сортировки}. Перевод с английского. Ред. Я. Я. Васина, В. Ю. Королев. Москва: Наука, 1983.
  
\end{thebibliography}