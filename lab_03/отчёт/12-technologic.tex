\section{Технологическая часть}
\subsection{Требования к программному обеспечению}

\hspace{1.20cm}
На вход программе подаются массив из целых чисел и целое число, которое будет искаться в нём.

На выход программа выдаёт индекс найденного в массиве элемента или -1 (на случай отсутствия искомого числа в массиве). Также в зависимости от выбранного пункта меню программа замеряет время работы алгоритмов и рисует получившиеся графики.

\subsection{Средства реализации}

\hspace{1.25cm}
В связи с опытом реализации подобных задач на Python, программа была реализована на этом языке программирования. Для замеров времени была использована функция \texttt{process\_time()} из библиотеки \texttt{time}, вычисляющая процессорное время \cite{process_time}.

\subsection{Реализации алгоритмов}

\hspace{1.25cm}
Ниже приведены реализации алгоритмов поиска элементов в массиве линейным способом (листинг \ref{lst:linear}) и бинарным (листинг \ref{lst:binary}) на Python.

\vspace{0.25cm}
\begin{lstlisting}[caption=реализация линейного алгоритма поиска в массиве, label=lst:linear]
def algo_search_order(arr, x):
    for i in range(len(arr)):
        if arr[i] == x:
            return i
    return -1
\end{lstlisting}

\vspace{0.25cm}
\begin{lstlisting}[caption=реализация бинарного алгоритма поиска в массиве, label=lst:binary]
def algo_search_bin(arr, x):
    arr = sorted(arr)
    left, right = 0, len(arr) - 1
    while left <= right:
        mid = left + (right - left) // 2
        if arr[mid] == x:
            return mid
        elif arr[mid] < x:
            left = mid + 1
        else:
            right = mid - 1
    return -1
\end{lstlisting}

\vspace{0.25cm}
\subsection{Тесты}

\hspace{1.25cm}
Для тестирования линейного и бинарного алгоритмов поиска элементов в массиве были составлены таблицы с входными данными (массив и искомый элемент), ожидаемым результатом (индексом) и полученным результатом от обоих способов.

\vspace{0.25cm}
\begin{table}[H]
    \centering
    \caption{Таблица тестов для алгоритмов поиска элементов в массиве}
    \renewcommand{\arraystretch}{1.5} % Увеличивает расстояние между строками
    \begin{tabular}{|c|c|c|c|c|c|}
        \hline
        \textbf{Массив} & \textbf{Искомый элемент} & \makecell{\textbf{Ожидание}\\линейный} & \makecell{\textbf{Ожидание}\\бинарный} & \makecell{\textbf{Результат}\\линейный} & \makecell{\textbf{Результат}\\бинарный}\\
        \hline
        [] & 1 & -1 & -1 & -1 & -1 \\
        \hline
        [1, 2, 3] & 4 & -1 & -1 & -1 & -1 \\
        \hline
        [1, 2, 3] & 1 & 0 & 0 & 0 & 0 \\
        \hline
        [1, 2, 3] & 2 & 1 & 1 & 1 & 1 \\
        \hline
        [1, 2, 3] & 3 & 2 & 2 & 2 & 2 \\
        \hline
        [3, 2, 1] & 1 & 2 & 0 & 2 & 0 \\
        \hline
    \end{tabular}
\end{table}

В ходе проведённого тестирования (с помощью pytest) ошибок в алгоритмах не выявлено (см листинг \ref{lst:pytest}).

\vspace{0.25cm}
\begin{lstlisting}[caption=тестирование алгоритмов с помощью pytest]
pytest
======================== test session starts ========================
platform win32 -- Python 3.11.9, pytest-8.2.2, pluggy-1.5.0
rootdir: L:\sem_5\\Algorithm_analysis\lab_03
plugins: Faker-28.4.1
collected 12 items                                                                                                                                                                

test_algo.py .....................................             [100%]

======================== 12 passed in 0.24s =========================
\label{lst:pytest}
\end{lstlisting}

\newpage