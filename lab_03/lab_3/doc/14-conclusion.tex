% Заключение
\begin{center}
    \MakeUppercase{\large Заключение}
\end{center}
\addcontentsline{toc}{section}{\MakeUppercase{Заключение}} % Добавляем в оглавление

В результате выполнения лабораторной работы были исследованы различные алгоритмы поска элемента в массиве (линейный и бинарный алгоритмы).

\vspace{0.25cm}
В частности:

\begin{enumerate}

\item рассмотрен линейный алгоритм поиска элемента в массиве;

\item рассмотрен бинарный алгоритм поиска элемента в массиве;

\item реализованы оба алгоритма поиска элемента в массиве на выбранном языке программирования;

\item сравнена эффективность этих алгоритмов на практике.

\end{enumerate}

В ходе лабораторной работы были рассмотрены, спроетированны и запрограммированы линейный и бинарный алгоритмы поиска элемента в массиве.

Были разработаны тесты для всех алгоритмов, учитывающие крайние случаи, ожидаемых результатов которых достигли все реализации.

Сравнения полученных программ показали, что бинарный алгоритм работает быстрее линейного, так как в большинстве случаев делает меньше сравнений, но, в связи с тем, что для бинарного поиска нужен отсортированный массив, такого выигрыша по времени не удастся добиться на несортированных данных.

\newpage