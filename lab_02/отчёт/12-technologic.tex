\section{Технологическая часть}
\subsection{Требования к программному обеспечению}

\hspace{1.15cm}
На вход программе подаются 2 матрицы из целых чисел.

На выход программа выдаёт матрицу --- произведение входных данных и код ошибки (на случай невозможности перемножить матрицы, если количество столбцов в первой не равно числу строк во второй). Также в зависимости от выбранного пункта меню программа замеряет время работы алгоритмов и рисует получившиеся графики.

\subsection{Средства реализации}

\hspace{1.25cm}
В связи с опытом реализации подобных задач на Python, программа была реализована на этом языке программирования. Для замеров времени была использована функция \texttt{process\_time()} из библиотеки \texttt{time}, вычисляющая процессорное время \cite{process_time}.

\subsection{Реализации алгоритмов}

\hspace{1.25cm}
Ниже приведены реализации алгоритмов умножения матриц стандартным способом (листинг \ref{lst:standart}), алгоритмом Винограда (листинг \ref{lst:Vinograd}) и оптимизированным алгоритмом Винограда (листинг \ref{lst:Vinograd_better}) на Python.

\vspace{0.25cm}
\begin{lstlisting}[caption=реализация стандартного алгоритма умножения матриц, label=lst:standart]
def algo_matrix_mult_standard(A: List[List[int]], B: List[List[int]]) -> Tuple[List[List[int]], bool]:
    rows1, cols1 = len(A), len(A[0])
    rows2, cols2 = len(B), len(B[0])
    C = [[0 for _ in range(cols2)] for _ in range(rows1)]
    if (cols1 != rows2):
        err_code = ERROR
    else:
        for i in range(rows1):
            for j in range(cols2):
                C[i][j] = 0
                for k in range(cols1):
                    C[i][j] = C[i][j] + A[i][k] * B[k][j]
        err_code = OK
    return C, err_code
\end{lstlisting}

\vspace{0.25cm}
\begin{lstlisting}[caption=реализация алгоритма Винограда умножения матриц, label=lst:Vinograd]
def algo_matrix_mult_Vinograd(A: List[List[int]], B: List[List[int]]) -> Tuple[List[List[int]], bool]:
    rows1, cols1 = len(A), len(A[0])
    rows2, cols2 = len(B), len(B[0])
    C = [[0 for _ in range(cols2)] for _ in range(rows1)]
    if (cols1 != rows2):
        err_code = ERROR
    else:
        # filling the array MulH
        MulH = [0 for _ in range(rows1)]
        for i in range(rows1):
            for k in range(int(cols1 / 2)):
                MulH[i] = MulH[i] + A[i][2*k] * A[i][2*k+1]
        # filling the array MulV
        MulV = [0 for _ in range(cols2)]
        for j in range(cols2):
            for k in range(int(cols1 / 2)):
                MulV[j] = MulV[j] + B[2*k][j] * B[2*k+1][j]
        # filling in the matrix C itself
        for i in range(rows1):
            for j in range(cols2):
                C[i][j] = - MulH[i] - MulV[j]
                for k in range(int(cols1 / 2)):
                    C[i][j] = C[i][j] + (A[i][2*k] + B[2*k+1][j]) * (A[i][2*k+1] + B[2*k][j])
        # if cols1 is odd (correction)
        if cols1 % 2 == 1:
            for i in range(rows1):
                for j in range(cols2):
                    C[i][j] = C[i][j] +  A[i][cols1-1] * B[cols1-1][j]
        err_code = OK
    return C, err_code
\end{lstlisting}

\vspace{0.25cm}
\begin{lstlisting}[caption=реализация оптимизированного алгоритма Винограда умножения матриц, label=lst:Vinograd_better]
def algo_matrix_mult_Vinograd_better(A: List[List[int]], B: List[List[int]]) -> Tuple[List[List[int]], bool]:
    rows1, cols1 = len(A), len(A[0])
    rows2, cols2 = len(B), len(B[0])
    C = [[0 for _ in range(cols2)] for _ in range(rows1)]
    if (cols1 != rows2):
        err_code = ERROR
    else:
        max_k = int(cols1 / 2)
        # filling the array MulH
        MulH = [0 for _ in range(rows1)]
        for i in range(rows1):
            if max_k != 0:
                MulH[i] = A[i][0] * A[i][1]
            for k in range(1, max_k):
                MulH[i] = MulH[i] + A[i][k << 1] * A[i][(k << 1) + 1]
        # filling the array MulV
        MulV = [0 for _ in range(cols2)]
        for j in range(cols2):
            if max_k != 0:
                MulV[j] = B[0][j] * B[1][j]
            for k in range(1, max_k):
                MulV[j] = MulV[j] + B[k << 1][j] * B[(k << 1) + 1][j]
        # filling in the matrix C itself
        for i in range(rows1):
            for j in range(cols2):
                C[i][j] = - MulH[i] - MulV[j]
                if cols1 != 1:
                    C[i][j] = C[i][j] + (A[i][0] + B[1][j]) * (A[i][1] + B[0][j])
                # if cols1 is odd (correction)
                if cols1 % 2 == 1:
                    C[i][j] = C[i][j] +  A[i][cols1-1] * B[cols1-1][j]
                for k in range(1, max_k):
                    C[i][j] = C[i][j] + (A[i][k << 1] + B[(k << 1) + 1][j]) * (A[i][(k << 1) + 1] + B[k << 1][j])
        err_code = OK
    return C, err_code
\end{lstlisting}

\vspace{0.25cm}
\subsection{Тесты}

\hspace{1.25cm}
Для тестирования всех алгоритмов умножения матриц были составлены тесты с входными данными (2 матрицы), ожидаемым результатом (матрицей --- произведением входных) и полученным результатом от всех способов (см таблицу \ref{table:tests}).

\vspace{0.25cm}
\begin{table}[H]
    \centering
    \caption{Таблица тестов для алгоритмов умножения матриц}
    \renewcommand{\arraystretch}{1.5} % Увеличивает расстояние между строками
    \begin{tabular}{|c|c|c|c|c|}
        \hline
        \textbf{Матрица A} & \textbf{Матрица B} & \textbf{Ожидание} & \makecell{\textbf{Результат}\\ \textbf{(код ошибки)}} & \makecell{\textbf{Результат}\\ \textbf{(матрица)}} \\
        \hline
        $\left( \right)$ & $\left( \right)$ & $\left( \right)$ & ERROR & $\left( \right)$ \\
        $\left( \right)$ & $\left( 1 \right)$ & $\left( \right)$ & ERROR & $\left( \right)$ \\
        $\left( \begin{matrix} 2 \end{matrix} \right)$ & $\left( \begin{matrix} 3 \end{matrix} \right)$ & $\left( \begin{matrix} 6 \end{matrix} \right)$ & OK & $\left( \begin{matrix} 6 \end{matrix} \right)$ \\
        $\left( \begin{matrix} 2 \end{matrix} \right)$ & $\left( \begin{matrix} 1 & 2 \end{matrix} \right)$ & $\left( \right)$ & ERROR & $\left( \right)$ \\
        $\left( \begin{matrix} 3 & 4 \end{matrix} \right)$ & $\left( \begin{matrix} 2 & 1 \\ 3 & 4 \end{matrix} \right)$ & $\left( \right)$ & ERROR & $\left( \right)$ \\
        $\left( \begin{matrix} 1 & 2 \\ 3 & 4 \end{matrix} \right)$ & $\left( \begin{matrix} 0 & -1 \\ 5 & -2 \end{matrix} \right)$ & $\left( \begin{matrix} 10 & -5 \\ 20 & -11 \end{matrix} \right)$ & OK & $\left( \begin{matrix} 10 & -5 \\ 20 & -11 \end{matrix} \right)$ \\
        $\left( \begin{matrix} 0 & -1 \\ 5 & -2 \end{matrix} \right)$ & $\left( \begin{matrix} 1 & 2 \\ 3 & 4 \end{matrix} \right)$ & $\left( \begin{matrix} -3 & -4 \\ -1 & 2 \end{matrix} \right)$ & OK & $\left( \begin{matrix} -3 & -4 \\ -1 & 2 \end{matrix} \right)$ \\
        $\left( \begin{matrix} 1 & 2 \\ 3 & 4 \\ 5 & 6 \end{matrix} \right)$ & $\left( \begin{matrix} -1 & 3 & 1 \\ 7 & 2 & 0 \end{matrix} \right)$ & $\left( \begin{matrix} 13 & 7 & 1 \\ 25 & 17 & 3 \\ 37 & 27 & 5 \end{matrix} \right)$ & OK & $\left( \begin{matrix} 13 & 7 & 1 \\ 25 & 17 & 3 \\ 37 & 27 & 5 \end{matrix} \right)$ \\
        \hline
    \end{tabular}
    \label{table:tests}
\end{table}

В ходе проведённого тестирования (с помощью pytest) ошибок в алгоритмах не выявлено (см листинг \ref{lst:pytest}).

\vspace{0.25cm}
\begin{lstlisting}[caption=тестирование алгоритмов с помощью pytest]
pytest
======================== test session starts ========================
platform win32 -- Python 3.11.9, pytest-8.2.2, pluggy-1.5.0
rootdir: L:\sem_5\\Algorithm_analysis\lab_02
plugins: Faker-28.4.1
collected 24 items                                                                                                                                                                

test_algo.py .....................................             [100%]

======================== 24 passed in 0.47s =========================
\end{lstlisting}
\label{lst:pytest}

\newpage