% Список использованных источников
\begin{center}
    \MakeUppercase{\large Список использованных источников}
\end{center}
\addcontentsline{toc}{section}{Список использованных источников} % Добавляем в оглавление

% Удаление "Список литературы" по умолчанию и получившихся пробелов
\renewcommand{\refname}{}
\vspace{-11mm}

\begin{thebibliography}{9}
    \bibitem{process_time} Python Documentation. \textit{time.process\_time() - Документация по стандартной библиотеке Python}. Дата обращения: 03 сентября 2024 г. [Электронный ресурс]. Доступно по адресу: \url{https://docs-python.ru/standart-library/modul-time-python/funktsija-process-time-modulja-time/}
    
    \bibitem{algolib} Algolib: коллекция алгоритмов. \textit{Умножение матриц}. Дата обращения: 19 сентября 2024 г. [Электронный ресурс]. Доступно по адресу: \url{https://algolib.narod.ru/Math/Matrix.html}
    
    \bibitem{nikitenko} Никитенко Е.В. Линейная алгебра и теория матриц. Учебное пособие для студентов всех форм обучения направления «Информатика и вычислительная техника» / Рубцовский индустриальный институт. – Рубцовск, 2022. – 56 с.
  
\end{thebibliography}