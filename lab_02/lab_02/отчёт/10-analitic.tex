% Основная часть
\section{Аналитическая часть}

\subsection{Базовые понятия матриц}

\hspace{1.25cm}
\textit{Матрицей $A$ размеров $m \times n$ называется совокупность $m \times n$ элементов из некоторого поля, расположенных в виде таблицы из $m$ строк и $n$ столбцов:}
    \[
    A = 
    \begin{pmatrix}
    a_{11} & a_{12} & \dots & a_{1n} \\
    a_{21} & a_{22} & \dots & a_{2n} \\
    \vdots & \vdots & \ddots & \vdots \\
    a_{m1} & a_{m2} & \dots & a_{mn}
    \end{pmatrix} = (a_{ij})_{m \times n},
    \]
    \textit{где $a_{ij}$ ($i = 1, 2, \dots, m$; $j = 1, 2, \dots, n$) — элементы матрицы, а $i$ указывает номер строки, а $j$ — номер столбца.}

    \textit{Матрица называется квадратной, если $m = n$, и число $n$ называется порядком матрицы.}
При этом число $n$ называется порядком матрицы.

У квадратной матрицы можно выделить главную и побочную диагонали:
\begin{itemize}
    \item Главная диагональ: $a_{11}, a_{22}, \dots, a_{nn}$;
    \item Побочная диагональ: $a_{n1}, a_{(n-1)2}, \dots, a_{1n}$.
\end{itemize}

Суммой $A + B$ двух матриц $A = (a_{ij})_{m \times n}$ и $B = (b_{ij})_{m \times n}$ называется матрица $C = (c_{ij})_{m \times n}$, где $c_{ij} = a_{ij} + b_{ij}$.

Произведением $\lambda \cdot A$ матрицы $A = (a_{ij})_{m \times n}$ на число $\lambda$ называется матрица $C = (c_{ij})_{m \times n}$, где $c_{ij} = \lambda \cdot a_{ij}$. \cite{nikitenko}

\subsection{Умножение матриц}

\hspace{1.25cm}
Произведением $A \cdot B$ матрицы $A = (a_{ij})_{m \times n}$ на матрицу $B = (b_{ij})_{n \times p}$ называется матрица $C = (c_{ij})_{m \times p}$, где
\[
c_{ij} = a_{i1}b_{1j} + a_{i2}b_{2j} + \dots + a_{in}b_{nj} = \sum_{k=1}^{n} a_{ik}b_{kj}.
\]

Элемент $c_{ij}$ равен сумме попарных произведений соответствующих элементов $i$-й строки матрицы $A$ и $j$-го столбца матрицы $B$.

Замечание. Перемножать можно только те матрицы, у которых число столбцов первой матрицы равно числу строк второй.

Свойства умножения матриц:
\begin{align*}
1) \ & \text{Не является коммутативным: } A \cdot B \neq B \cdot A \text{ в общем случае}; \\
2) \ & \text{Ассоциативность: } (A \cdot B) \cdot C = A \cdot (B \cdot C); \\
3) \ & \text{Дистрибутивность: } (A + B) \cdot C = A \cdot C + B \cdot C; \\
4) \ & \text{Для любой квадратной матрицы } A \text{ порядка } n \text{ справедливо: } A \cdot E = E \cdot A = A,
\end{align*}
где $E$ — единичная матрица порядка $n$.
\cite{nikitenko}

\subsection{Умножение матриц по Винограду}

\hspace{1.25cm}
Если посмотреть на результат умножения двух матриц, то видно, что каждый элемент представляет собой скалярное произведение соответствующих строки и столбца исходных матриц. Такое умножение допускает предварительную обработку, позволяющую часть работы выполнить заранее.

Рассмотрим два вектора $V = (v_1, v_2, v_3, v_4)$ и $W = (w_1, w_2, w_3, w_4)$. Их скалярное произведение равно:
\[
V \cdot W = v_1 w_1 + v_2 w_2 + v_3 w_3 + v_4 w_4.
\]

Это равенство можно переписать в виде:
\[
V \cdot W = (v_1 + w_2)(v_2 + w_1) + (v_3 + w_4)(v_4 + w_3) - v_1 v_2 - v_3 v_4 - w_1 w_2 - w_3 w_4.
\]

Эквивалентность двух выражений можно легко проверить. Кажется, что второе выражение задает больше работы, чем первое: вместо четырёх умножений мы видим шесть, а вместо трёх сложений — десять. Менее очевидно, что правое выражение допускает предварительную обработку: его части можно вычислить заранее и сохранить для каждой строки первой матрицы и для каждого столбца второй.

На практике это означает, что после предварительной обработки остаётся выполнить два умножения и семь сложений.
\cite{algolib}

\newpage