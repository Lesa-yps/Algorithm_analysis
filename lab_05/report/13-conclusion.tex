% Заключение
\begin{center}
    \MakeUppercase{\large Заключение}
\end{center}
\addcontentsline{toc}{section}{\MakeUppercase{Заключение}} % Добавляем в оглавление

В результате выполнения лабораторной работы были получены навыки организации параллельных вычислений по конвейерному принципу.

\vspace{0.25cm}
В частности:

\begin{enumerate}

\item был проведён анализ предметной области;

\item разработан алгоритм обработки данных;

\item создано ПО реализующее разработанный алгоритм;

\item исследованы характеристики созданного ПО.

\end{enumerate}

В ходе лабораторной работы был рассмотрен, спроектирован и запрограммирован алгоритм парсинга страниц и записи данных в базу с помощью нативных потоков по конвейерному принципу.

Проведённые замеры времени обработки данных через потоки на разных стадиях показали, что чтение данных и их обработка на первых двух стадиях происходит быстрее, чем запись этих данных в хранилище. В связи с этим время нахождения в очереди к потоку загружающему данные в базу также больше, чем в очереди к потоку, занимающемуся выборкой данных.

Таким образом, лабораторная работа позволила не только освоить принципы организации параллельных вычислений по конвейерному принципу, но и на практике выявить узкие места в процессах обработки данных. Особенно это касается этапа записи данных в хранилище, который оказался более времязатратным по сравнению с другими этапами. Кроме того, проведённые эксперименты продемонстрировали преимущества использования многозадачности для ускорения обработки данных, что делает данное решение перспективным для применения в реальных системах, требующих обработки больших объёмов информации в режиме реального времени.

\newpage