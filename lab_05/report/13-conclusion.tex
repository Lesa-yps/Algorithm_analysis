% Заключение
\begin{center}
    \MakeUppercase{\large Заключение}
\end{center}
\addcontentsline{toc}{section}{\MakeUppercase{Заключение}} % Добавляем в оглавление

В результате выполнения лабораторной работы были получены навыки организации параллельных вычислений на основе нативных потоков.

\vspace{0.25cm}
В частности:

\begin{enumerate}

\item был проведён анализ предметной области;

\item разработан алгоритм обработки данных;

\item создано ПО реализующее разработанный алгоритм;

\item исследованы характеристики созданного ПО.

\end{enumerate}

В ходе лабораторной работы был рассмотрен, спроектированн и запрограммирован алгоритм загрузки страниц с помощью нативных потоков.

Проведённые замеры времени выполнения программного обеспечения при использовании различного числа потоков показали значительное ускорение обработки данных. Эти результаты подтверждают эффективность подхода к параллельной обработке данных и показывают, что увеличение числа потоков приводит к улучшению производительности, особенно при наличии достаточного количества логических ядер. Наблюдаемое время выполнения на уровне 3.79 секунд при использовании 64 потоков также указывает на наличие эффекта насыщения, что подчёркивает необходимость оптимального выбора количества потоков для достижения максимальной производительности.

Таким образом, лабораторная работа не только обеспечила практические навыки в области работы с нативными потоками, но и подтвердила теоретические знания о преимуществах и ограничениях параллельных вычислений.

\newpage