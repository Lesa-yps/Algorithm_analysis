% Тестирование
\section{Тестирование}

\hspace{1.25cm}
Выполнено тестирование реализованной основной части программы по методологии чёрного ящика. В таблице~\ref{tab:tests} представлено описание тестов. Все тесты пройдены успешно.

\begin{table}[h!]
    \centering
    \caption{Функциональные тесты}
    \label{tab:tests}
    \begin{tabularx}{\textwidth}{|c|X|X|c|}
        \hline
        № теста & Входные данные                                  & Ожидаемые выходные данные                                                  & Успешность \\ \hline
        1       & Файл `links.txt` не существует                  & Программа выводит ошибку: "Не удалось открыть файл"                        & Ошибка (1) \\ \hline
        2       & Файл `links.txt` пуст                           & Программа выводит: "Нет доступных URL для обработки."                      & Ошибка (2) \\ \hline
        3       & Файл `links.txt` содержит 1 URL                 & Программа успешно загружает страницу на 1 потоке и выводит время загрузки  & Успех      \\ \hline
        4       & Файл `links.txt` содержит 1000 URL              & Программа корректно загружает все страницы в многопоточном режиме          & Успех      \\ \hline
        5       & Недопустимые символы в URL                      & Программа корректно заменяет недопустимые символы в именах файлов          & Успех      \\ \hline
        6       & Ввод числа потоков больше числа URL             & Программа выводит ошибку и запрашивает корректное число потоков            & Ошибка     \\ \hline
        7       & Ввод несуществующего пункта меню                & Программа выводит: "Неверный пункт меню. Попробуйте снова."                & Ошибка     \\ \hline
        8       & Файл уже обработан, повторный запуск загрузки   & Программа корректно завершает работу без повторной загрузки                & Успех      \\ \hline
        9       & URL превышает максимальную длину                & Программа корректно обрабатывает или игнорирует длинный URL                & Успех      \\ \hline
        10      & Папка `data` уже существует                     & Программа сохраняет страницы в существующую папку без ошибок               & Успех      \\ \hline
    \end{tabularx}
\end{table}

Файл `links.txt` со списком ссылок для обработки.

Данные выводятся в папку `data`.
