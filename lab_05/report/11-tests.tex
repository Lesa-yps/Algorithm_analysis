% Тестирование
\section{Тестирование}

\hspace{1.25cm}
Выполнено тестирование реализованной основной части программы по методологии чёрного ящика. В таблице~\ref{tab:tests} представлено описание тестов. Все тесты пройдены успешно.

\begin{table}[h!]
    \centering
    \caption{Функциональные тесты}
    \label{tab:tests}
    \begin{tabularx}{\textwidth}{|c|X|X|c|}
        \hline
        № теста & Входные данные                                  & Ожидаемые выходные данные                                                  & Успешность теста \\ \hline
        1       & Папка с файлами, структура данных корректна    & Заявки успешно добавлены в очередь 1, корректная обработка данных           & Успешно    \\ \hline
        2       & Пустая папка                                    & Нет заявок для обработки, программа завершает работу без ошибок           & Успешно    \\ \hline
        3       & Некорректный формат данных в файле (HTML)        & Программа игнорирует файл, продолжает обработку остальных задач           & Успешно    \\ \hline
        4       & Папка с файлами, корректная структура данных   & Заявки корректно проходят через очереди, данные записываются в БД         & Успешно    \\ \hline
        5       & Большое количество файлов (параллельная обработка) & Все файлы обрабатываются в многозадачном режиме, результаты записываются в БД  & Успешно    \\ \hline
        6       & Проблемы с подключением к БД                    & Программа выводит ошибку подключения и завершает работу                    & Успешно    \\ \hline
        7       & Недостаток памяти (искусственно)                & Программа корректно обрабатывает ошибку выделения памяти                   & Успешно    \\ \hline
        8       & Неожиданное завершение потока                    & Программа завершает работу с выводом ошибок при некорректной работе потока  & Успешно    \\ \hline
    \end{tabularx}
\end{table}
