% Введение
% Центрируем заголовок и делаем его заглавными буквами
\begin{center}
    \MakeUppercase{Введение}
\end{center}
\addcontentsline{toc}{section}{\MakeUppercase{Введение}} % Добавляем в оглавление с заглавными буквами

В современных вычислительных системах организация параллельных вычислений играет ключевую роль в повышении производительности программного обеспечения. Одним из эффективных методов структурирования параллельных вычислений является конвейерный принцип, при котором задача делится на несколько последовательных этапов. Каждый этап выполняется в отдельном потоке, позволяя системе одновременно обрабатывать разные части задачи на разных уровнях конвейера. Такой подход особенно полезен для задач с последовательными стадиями обработки данных, поскольку он позволяет распределить выполнение между несколькими ядрами процессора, минимизируя время простоя и повышая общую производительность программы.
\vspace{0.25cm}

Цель работы: получение навыка организации параллельных вычислений по конвейерному принципу.

\vspace{0.25cm}
Для достижения этой цели были поставлены следующие задачи:

\begin{enumerate}

\item анализ предметной области;
\item разработка алгоритма обработки данных;
\item создание ПО реализующего разработанный алгоритм;
\item исследование характеристик созданного ПО.

\end{enumerate}

% Входные и выходные данные
\section{Входные и выходные данные}

\hspace{1.25cm}
Входными данными для ПО является папка data, содержащая html файлы со сказанными кулинарными страницами. Выходные данные --- база данных с таблицами, содаржащими рецепты, ингредиенты для них и шаги по приготовлению. Также программа делает замеры времени для получения максимального, минимального, среднего и медианного времён нахождения в очередях 2 и 3 и обработки на трёх стадиях.

% Преобразование входных данных в выходные
\section{Преобразование входных данных в выходные}

\hspace{1.25cm}
Программа запускает 5 потоков, каждый из которых выполняет свою последовательную часть обработки заявки. Первый поток создаёт задачи с путями к файлам из папки data и помещает их в первую очередь.  Второй поток берёт заявки из первой очереди, читает файлы и дополняет заявки недостающими данными о рецептах, затем помещает задачу во вторую очередь. Третий поток берёт заявку из второй очереди, очищает выбранные из файла данные от html символов и помещает задачу в третью очередь. Четвёртый поток берёт заявки из третьей очереди, записывает данные из неё в таблицы базы данных и помещает задачу в четвёртую очередь. Пятый поток берёт заявки из четвёртой очереди и вычисляет максимальное, минимальное, среднее и медианное времена нахождения в очередях 2 и 3 и обработки на предыдущих трёх стадиях, попутно уничтожая заявки. Таким образом программа реализует параллельные вычисления по конвейерному принципу.~\cite{appmaster}