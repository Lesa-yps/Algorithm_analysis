% Список использованных источников
\addcontentsline{toc}{section}{\MakeUppercase{Список использованных источников}} % Добавляем в оглавление

\renewcommand{\refname}{\begin{center}\MakeUppercase{Список использованных источников}\end{center}}
%\vspace{-11mm}

\begin{thebibliography}{9}

\bibitem{tetraquark}
\textit{Методы парсинга данных на C++ и Python.} Tetraquark. [Электронный ресурс]. URL: \url{https://tetraquark.ru/archives/47} (дата обращения: 22.10.2024).

\bibitem{metanit}
\textit{Парсинг HTML и XML в C\#.} Metanit. [Электронный ресурс]. URL: \url{https://metanit.com/c/tutorial/11.1.php} (дата обращения: 22.10.2024).

\bibitem{pythonru}
\textit{Парсинг на Python с библиотекой Beautiful Soup.} PythonRu. [Электронный ресурс]. URL: \url{https://pythonru.com/biblioteki/parsing-na-python-s-beautiful-soup} (дата обращения: 22.10.2024).

\bibitem{thecode}
\textit{Парсинг с помощью Python.} The Code. [Электронный ресурс]. URL: \url{https://thecode.media/parsing-2/} (дата обращения: 23.10.2024).
  
\end{thebibliography}