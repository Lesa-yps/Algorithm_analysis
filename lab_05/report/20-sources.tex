% Список использованных источников
\addcontentsline{toc}{section}{\MakeUppercase{Список использованных источников}} % Добавляем в оглавление

\renewcommand{\refname}{\begin{center}\MakeUppercase{Список использованных источников}\end{center}}
%\vspace{-11mm}

\begin{thebibliography}{9}

\bibitem{tetraquark}
\textit{Методы парсинга данных на C++ и Python.} Tetraquark. [Электронный ресурс]. URL: \url{https://tetraquark.ru/archives/47} (дата обращения: 22.10.2024).

\bibitem{appmaster}
AppMaster, \textit{Конвейерное программирование}, доступно по ссылке: \url{https://appmaster.io/ru/glossary/konveiernoe-programmirovanie}, дата обращения: 14 ноября 2024.
  
\end{thebibliography}