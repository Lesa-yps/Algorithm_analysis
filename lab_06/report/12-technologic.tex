\section{Технологическая часть}
\subsection{Требования к программному обеспечению}

\hspace{1.20cm}
\underline{Входные данные:}\\
Программа принимает файл, содержащий матрицу расстояний между городами, где значение -1 означает отсутствие пути. Также задается список направлений рек  и текущий сезон, влияющий на затраты на прохождение рек (зима или лето). Для решения задачи муравьиным алгоритмом дополнительно вводятся следующие параметры:

\begin{enumerate}[label=\arabic*)]
  \item alpha -- параметр влияния длины пути;
  \item beta -- параметр влияния феромона;
  \item koef\_evap -- коэффициент испарения феромона;
  \item days -- количество дней гуляния муравьёв.
\end{enumerate}

\underline{Выходные данные:}\\
На выходе программа предоставляет кратчайший незамкнутый маршрут обхода всех городов и его длину, вычисленные алгоритмом полного перебора или муравьиным алгоритмом.

\subsection{Средства реализации}

\hspace{1.25cm}
В связи с опытом реализации подобных задач на Python, программа была реализована на этом языке программирования.

\subsection{Реализации алгоритмов}

\hspace{1.25cm}
Ниже приведена реализация алгоритма коммивояжёра полным перебором (листинг~\ref{lst:full}) и муравьиным алгоритмом (листинг~\ref{lst:main_ant_algo}) на Python.

\vspace{0.25cm}
\begin{lstlisting}[caption=реализации алгоритма коммивояжёра полным перебором, label=lst:full]
def Algo_full_search_paths(matrix_paths, river_direct_arr, season):
    n = len(matrix_paths)
    varios_paths_id_arr = [i for i in range(n)]
    all_combinations_paths = list()

    for path in it.permutations(varios_paths_id_arr):
        all_combinations_paths.append(list(path))

    min_len_path = float("+inf")
    min_ind_path = -1

    for j in range(len(all_combinations_paths)):
        is_path_exist = True
        path = all_combinations_paths[j]

        len_path = 0
        i = 0
        while is_path_exist and i < (n - 1):
            ind_start_city = path[i]
            ind_finish_city = path[i + 1]
            len_erge = matrix_paths[ind_start_city][ind_finish_city]
            if len_erge < 0:
                is_path_exist = False
            else:
                len_path += effect_rivers_seasons(len_erge, ind_start_city, ind_finish_city, river_direct_arr, season)
                i += 1
        
        if is_path_exist and len_path < min_len_path:
            min_len_path = len_path
            min_ind_path = j

    if min_ind_path == -1:
        res_path = list()
    else:
        res_path = all_combinations_paths[min_ind_path]
    
    return min_len_path, res_path
\end{lstlisting}

\vspace{0.25cm}


\begin{lstlisting}[caption=реализации алгоритма коммивояжёра муравьиным алгоритмом, label=lst:main_ant_algo]
def Algo_ant_search_paths(matrix_edges, river_direct_arr, season, alpha, beta, koef_evap, days):
    count_places = len(matrix_edges)
    min_path = list()
    min_len_path = float("inf")
    avg_len_edge = calc_avg_len_edge(matrix_edges, count_places, river_direct_arr, season)
    pherom_matrix = init_pherom_matrix(count_places)
    count_ants = count_places
    visibil_matrix = init_visibil_matrix(matrix_edges, count_places, river_direct_arr, season)

    for day_num in range(days):
        if day_num != 0 and day_num % CHANGE_SEASON == 0:
            season = (season + 1) % 2
            visibil_matrix = init_visibil_matrix(matrix_edges, count_places, river_direct_arr, season)
            avg_len_edge = calc_avg_len_edge(matrix_edges, count_places, river_direct_arr, season)
        
        paths_for_all_ants = init_paths_for_all_ants(count_ants)

        for ant_num in range(count_ants):
            flag_deadlock = False
            while len(paths_for_all_ants[ant_num]) != count_places:
                possibil_places_visit_arr = find_posibls_of_visit_places(pherom_matrix, visibil_matrix, paths_for_all_ants, count_places, ant_num, alpha, beta)
                if all(value == 0 for value in possibil_places_visit_arr):
                    flag_deadlock = True
                    break
                chosen_place = choose_next_place(possibil_places_visit_arr)
                paths_for_all_ants[ant_num].append(chosen_place - 1)

            if flag_deadlock:
                continue

            len_path = calc_len_path(matrix_edges, paths_for_all_ants[ant_num], river_direct_arr, season)
            if len_path < min_len_path:
                min_len_path = len_path
                min_path = paths_for_all_ants[ant_num]

        pherom_matrix = update_pherom_matrix(matrix_edges, count_places, count_ants, paths_for_all_ants, pherom_matrix, avg_len_edge, koef_evap, river_direct_arr, season)
    
    return min_len_path, min_path
\end{lstlisting}


\vspace{0.25cm}
\subsection{Тесты}

\hspace{1.25cm}
Для тестирования муравьиного алгоритма (МА) и алгоритма полного перебора (АПП) были составлены таблицы с входными данными, ожидаемыми результатами и успехом выполнения каждого алгоритма. Входные данные включают матрицу расстояний между точками, направления рек и сезон (0 -- лето, 1 -- зима).

\vspace{0.25cm}
\begin{table}[H]
    \centering
    \caption{Таблица тестов для алгоритмов поиска кратчайшего пути}
    \renewcommand{\arraystretch}{1.5} % Увеличивает расстояние между строками
    \begin{tabular}{|c|c|c|c|}
        \hline
        \textbf{Входные данные} & \textbf{Ожидаемый результат} & \textbf{Прохождение АПП} & \textbf{Прохождение МА} \\
        \hline
        \makecell{0 3 -1 7 \\ 3 0 5 6 \\ 10 -1 0 5 \\ -1 -1 2 0 \\ --- \\ 0 2 \\ 1 3 \\ --- \\ 0} & \makecell{Путь = [0, 1, 3, 2] \\ Длина = 8.0} & \textbf{Успех} & \textbf{Успех} \\
        \hline
        \makecell{0 -1 920 600 500 \\ 850 0 500 700 170 \\ -1 500 0 -1 800 \\ 600 700 50 0 450 \\ 600 100 800 -1 0 \\ --- \\ 0 3 \\ --- \\ 0} & \makecell{Путь = [0, 3, 2, 1, 4] \\ Длина = 1020.0} & \textbf{Успех} & \textbf{Успех} \\
        \hline
    \end{tabular}
\end{table}

В ходе проведённого тестирования ошибок в алгоритмах не выявлено.

\newpage