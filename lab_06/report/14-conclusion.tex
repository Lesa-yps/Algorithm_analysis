% Заключение
\begin{center}
    \MakeUppercase{\large Заключение}
\end{center}
\addcontentsline{toc}{section}{\MakeUppercase{Заключение}} % Добавляем в оглавление

В результате выполнения лабораторной работы были исследованы два подхода к решению задачи коммивояжёра: метод полного перебора и муравьиный алгоритм.

\vspace{0.25cm}
В частности:

\begin{enumerate}[label=\arabic*)]

\item рассмотрен метод полного перебора;

\item рассмотрен муравьиный алгоритм;

\item реализованы оба алгоритма решения задачи коммивояжёра на выбранном языке программирования;

\item проведена параметризация муравьиного алгоритма по четырём параметрам и оценено качество решений для серии графов.

\end{enumerate}

В ходе лабораторной работы были рассмотрены, спроектированы и запрограммированы алгоритм полного перебора и муравьиный алгоритм решения задачи коммивояжёра.

Были разработаны тесты для всех алгоритмов, ожидаемых результатов которых достигли все реализации.

Проведённые исследования по запуску трёх графов на серии различных комбинаций входных параметров показал уменьшение отличия результата муравьиного алгоритма от эталонного ответа, данного полным перебором, с уменьшение числа рёбер графа и с увеличением числа дней жизни муравьиной колонии.

\newpage