% Основная часть
\section{Аналитическая часть}

\subsection{Задача коммивояжёра}

\hspace{1.25cm}
Задача коммивояжёра -- одна из самых известных оптимизационных задач. Её цель заключается в нахождении самого выгодного маршрута (кратчайшего, самого быстрого, наиболее дешёвого), проходящего через все заданные точки (пункты, города) по одному разу. 

\textbf{Условия задачи} должны включать критерий выгодности маршрута (например, минимальная длина, время, стоимость или их комбинация), а также исходные данные в виде матрицы затрат (расстояний, стоимости, времени и прочие) при перемещении между пунктами. 

Особенность задачи заключается в том, что она просто формулируется, и хорошие приближённые решения можно найти относительно легко. Однако поиск действительно оптимального маршрута для большого набора данных требует значительных вычислительных ресурсов.

\underline{Математическая модель задачи}

Для решения задачи коммивояжёра её можно представить как математическую модель. Исходные условия записываются в формате:
\begin{enumerate}[label=\arabic*)]
    \item \textbf{матрицы} -- таблицы, где строки соответствуют городам отправления, столбцы -- городам прибытия, а элементы матрицы представляют расстояния, время или стоимость перемещения;
    \item \textbf{графа} -- схемы, состоящей из вершин (города) и рёбер (пути между городами), где длина рёбер соответствует расстояниям.
\end{enumerate}

\underline{Классификация задачи коммивояжёра}

\textbf{По симметричности рёбер графа:}
\begin{enumerate}[label=\arabic*)]
    \item \textbf{симметричная задача} (\textit{Symmetric TSP}): длина пути между любыми двумя городами одинакова в обоих направлениях (неориентированный граф);
    \item \textbf{асимметричная задача} (\textit{Asymmetric TSP}): длина пути между городами может различаться в зависимости от направления (ориентированный граф).
\end{enumerate}

\textbf{По замкнутости маршрута:}
\begin{enumerate}[label=\arabic*)]
    \item \textbf{замкнутая задача:} поиск кратчайшего пути через все города с возвратом в исходную точку;
    \item \textbf{незамкнутая задача:} поиск кратчайшего пути через все города без обязательного возврата в исходную точку.~\cite{galautdinov}
\end{enumerate}

\subsection{Методы решения задачи коммивояжёра}

\hspace{1.25cm}
Существует множество методов, различающихся инструментарием, точностью и вычислительной сложностью. В этом разделе будут подробнее рассмотрены метод полного перебора и муравьиный алгоритм.

\subsubsection{Метод полного перебора}

\hspace{1.25cm}
Метод полного перебора заключается в построении всех возможных маршрутов для заданного множества городов. Для каждого маршрута вычисляется сумма весов рёбер, составляющих путь, и выбирается маршрут с минимальной общей длиной.

\underline{Алгоритм метода полного перебора:}
\begin{enumerate}[label=\arabic*)]
    \item построить все возможные перестановки городов;
    \item для каждой перестановки вычислить длину маршрута как сумму весов соответствующих рёбер;
    \item выбрать маршрут с минимальной длиной.
\end{enumerate}

\underline{Преимущества:}
\begin{enumerate}[label=\arabic*)]
    \item гарантированное нахождение оптимального решения;
    \item простота реализации.
\end{enumerate}

\underline{Недостатки:}
\begin{enumerate}[label=\arabic*)]
    \item высокая вычислительная сложность $O(n!)$, где $n$ -- количество городов;
    \item невозможность применения для больших графов из-за экспоненциального роста числа перестановок.
\end{enumerate}

\underline{Пример:}
Приведён граф с четырьмя городами. Матрица весов рёбер задана следующим образом:
\begin{equation*}
\begin{bmatrix}
    0 & 10 & 15 & 20 \\
    10 & 0 & 35 & 25 \\
    15 & 35 & 0 & 30 \\
    20 & 25 & 30 & 0
\end{bmatrix}
\end{equation*}
Все возможные маршруты: $(1 \to 2 \to 3 \to 4 \to 1)$, $(1 \to 3 \to 2 \to 4 \to 1)$ и другие. Для каждого маршрута вычисляем сумму весов рёбер и выбираем минимальную.


\subsubsection{Муравьиный алгоритм}

\hspace{1.25cm}
Муравьиный алгоритм основан на моделировании поведения реальной муравьиной колонии. Этот метод использует взаимодействие между муравьями через химическое вещество -- феромон, который откладывается на пройденных маршрутах. Феромон позволяет учитывать опыт других муравьёв при выборе следующего пути.

\underline{Основные элементы алгоритма:}
\begin{enumerate}[label=\arabic*)]
    \item \textbf{феромон:} определяет предпочтительность выбора маршрута, увеличиваясь на коротких путях;
    \item \textbf{видимость:} локальная информация, обратная расстоянию между городами $\eta_{ij} = 1 / d_{ij}$;
    \item \textbf{память:} список уже посещённых городов, предотвращающий повторное посещение.
\end{enumerate}

\underline{Этапы алгоритма:}
\begin{enumerate}[label=\arabic*)]
    \item инициализация параметров алгоритма: количество муравьёв, начальные значения феромона, параметры $\alpha$ (влияние феромона) и $\beta$ (влияние видимости);
    \item для каждого муравья на текущей итерации выполняется выбор следующего города на основе вероятности перехода:
    \begin{equation*}
    P_{ij}^k = \frac{(\tau_{ij})^\alpha (\eta_{ij})^\beta}{\sum_{l \in \text{доступные}} (\tau_{il})^\alpha (\eta_{il})^\beta},
    \end{equation*}
    где $\tau_{ij}$ -- уровень феромона на ребре, $\eta_{ij}$ -- видимость, $\alpha$ и $\beta$ -- параметры алгоритма;
    \item после завершения всех маршрутов обновляются значения феромона по формуле:
    \begin{equation*}
    \tau_{ij}(t+1) = (1 - \rho) \tau_{ij}(t) + \Delta \tau_{ij},
    \end{equation*}
    где $\rho$ -- коэффициент испарения, а $\Delta \tau_{ij}$ рассчитывается по формуле:
    \begin{equation*}
    \Delta \tau_{ij} = \sum_{k=1}^{m} \frac{Q}{L_k},
    \end{equation*}
    где $L_k$ -- длина маршрута муравья $k$, $Q$ -- регулируемый параметр;
    \item проверяется условие остановки (например, достижение заданного числа итераций), и в случае выполнения выводится лучший маршрут.
\end{enumerate}

\underline{Преимущества:}
\begin{enumerate}[label=\arabic*)]
    \item эффективность для больших графов;
    \item гибкость за счёт параметров $\alpha$, $\beta$ и $\rho$.
\end{enumerate}

\underline{Недостатки:}
\begin{enumerate}[label=\arabic*)]
    \item возможность попадания в локальные минимумы;
    \item сложность настройки параметров.~\cite{bachelor_tsp}
\end{enumerate}


\subsection{Выводы}

\hspace{1.25cm}
В данной аналитической части были рассмотрены два подхода к решению задачи коммивояжёра: метод полного перебора и муравьиный алгоритм.

Метод полного перебора, несмотря на его вычислительную сложность, предоставляет эталонное решение задачи, но применим только для небольших графов.

Муравьиный алгоритм, напротив, обеспечивает высокую производительность и подходит для задач с большим числом городов, хотя требует тонкой настройки параметров для достижения оптимальных результатов.

\newpage